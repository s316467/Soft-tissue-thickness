\documentclass{article_saj}
\pagestyle{myheadings}
\usepackage{graphicx,saj,multicol,subeqnarray}
\usepackage{natbib}
\usepackage{float}
\usepackage{xcolor}
\usepackage{widetext}
\usepackage{url}
\usepackage{bm}
\usepackage{tikz} % for checkmark
\usepackage{pifont} % for xmark
\usepackage{amsfonts}
\usepackage{amssymb}
\usepackage{amsmath,upgreek}
\usepackage{titlesec}
\def\tg{\mathop{\rm tg}\nolimits}
\def\arctg{\mathop{\rm arctg}\nolimits}

\def\point#1{\hbox{\setbox7=\hbox to0.6em{\hfil.\hfil}%
\setbox8=\hbox to0.5em{\hfil$^{#1}$\hfil}%
\box7\kern-0.5em\box8}}

\def\pointmin#1{\hbox{\setbox2=\hbox to0.8em{\hfil.\hfil}%
\setbox3=\hbox to0.6em{\hfil$^{#1}$\hfil}%
\box2\kern-.7em\box3}}

%for non integer numbers for years, days, hours, minutes and seconds

\def\yyy{\point{\mathrm{y}}}
\def\ddd{\point{\mathrm{d}}}
\def\hhh{\point{\mathrm{h}}}
\def\mmm{\pointmin{\mathrm{m}}\kern.15em}
\def\sss{\point{\mathrm{s}}}

%for non integer numbers for arc degrees, arc minutes and arc seconds

\def\oo{\point{\circ}}
\def\lll{\point{\prime}}
\def\uu{\point{\prime\prime}}

%for integer numbers for arc degrees, arc minutes and arc seconds

\def\OO{$^\circ$}
\def\LLL{$^\prime$}
\def\UU{$^{\prime\prime}$}
 

\titlelabel{\thetitle.\quad}
\definecolor{xlinkcolor}{cmyk}{1,0.6,0,0}
\usepackage[bookmarks=false,         % show bookmarks bar?
     pdfnewwindow=true,      % links in new window
     colorlinks=true,    % false: boxed links; true: colored links
     linkcolor=xlinkcolor,     % color of internal links
     citecolor=xlinkcolor,     % color of links to bibliography
     filecolor=xlinkcolor,  % color of file links
     urlcolor=xlinkcolor,      % color of external links
final=true
]{hyperref}

% Papertype can be "Invited Review", "Original Scientific Paper",
% "Preliminary report" or "Professional paper"

\def\papertype{\ \hfill\ Professional paper}

\setcounter{publyear}{2024}
\setcounter{page}{1}
\setcounter{firstpage}{1}
\setcounter{lastpage}{5}

\citestyle{kluwer}%


\begin{document}
%
\parindent=.5cm
\baselineskip=3.8truemm
\columnsep=.5truecm
%
\newenvironment{lefteqnarray}{\arraycolsep=0pt\begin{eqnarray}}
{\end{eqnarray}\protect\aftergroup\ignorespaces}
\newenvironment{lefteqnarray*}{\arraycolsep=0pt\begin{eqnarray*}}
{\end{eqnarray*}\protect\aftergroup\ignorespaces}
\newenvironment{leftsubeqnarray}{\arraycolsep=0pt\begin{subeqnarray}}
{\end{subeqnarray}\protect\aftergroup\ignorespaces}
%

% Runningtitle


\begin{strip}

{\ }

\vskip-1cm

\publ

\type

{\ }

% Title

\title{Soft-Tissue Thickness}

\author{
Amato Federica\\
Politecnico di Torino\\
{\tt\small s317496@studenti.polito.it}
\and
Di Iorio Matteo\\
Politecnico di Torino\\
{\tt\small s316606@studenti.polito.it}
\and
Ferrigno Antonio\\
Politecnico di Torino\\
{\tt\small s316467@studenti.polito.it}
\and
Figliolino Giulio\\
Politecnico di Torino\\
{\tt\small s317510@studenti.polito.i}
\and
Gjikopulli Fatjona\\
Politecnico di Torino\\
{\tt\small s310275@studenti.polito.it}
}

\maketitle

% Abstract

\summary{This project aims to evaluate the soft-tissue thickness (STT) on the face, particularly at specific landmarks or fiducial points, using DICOM data. The study will be contextualized within a relevant application and will include statistical analyses to assess differences based on gender, surgical intervention, and body mass index (BMI). The accuracy of soft-tissue thickness evaluation will be compared with the current precision standards in the literature. The results obtained will be benchmarked against established data to ensure reliability and validity.}


% Keywords (see keywords.pdf file)

\keywords{Soft-tissue thickness -- DICOM data -- Facial landmarks -- Statistical analysis -- Gender differences -- Surgical intervention -- Body mass index -- Accuracy comparison}

\end{strip}

\tenrm

% Sections

\section{INTRODUCTION}

\indent

The evaluation of soft-tissue thickness (STT) is crucial in various medical and clinical applications, particularly in craniofacial studies. This project aims to measure the thickness of soft tissues on the face at specific landmarks or fiducial points using DICOM data. These measurements are essential for applications such as surgical planning, forensic analysis, and anthropometric studies.

\renewcommand{\thefootnote}{\arabic{footnote}}

The primary objectives of this study include:

\parindent=.7cm

\par\hang\textindent{(i)} Utilizing DICOM data to accurately measure soft-tissue thickness at defined facial landmarks.

\par\hang\textindent{(ii)} Contextualizing these measurements within a specific application, ensuring relevance and practical utility.

\par\hang\textindent{(iii)} Performing statistical analyses to assess variations in STT based on gender, surgical intervention, and body mass index (BMI).

\par\hang\textindent{(iv)} Ensuring that the accuracy of STT measurements aligns with current standards in the literature, facilitating reliable comparisons and validations of results.

\parindent=.5cm

Soft-tissue thickness measurements have significant implications in both clinical and research settings. Accurate assessments can enhance the precision of surgical interventions, improve forensic reconstructions, and contribute to the understanding of facial morphology variations across different populations. This project leverages advanced imaging techniques and robust statistical methods to provide comprehensive insights into the variations of soft-tissue thickness, thereby contributing to the existing body of knowledge and offering practical solutions in medical and anthropometric applications.

The methodology involves the use of high-resolution DICOM data, which allows for precise and reproducible measurements. By incorporating statistical analyses, this study aims to identify significant differences in STT across different demographics and clinical conditions. The results will be benchmarked against established data to ensure reliability and validity, thereby enhancing the credibility and applicability of the findings in real-world scenarios.


\section{RELATED WORK}

In this section, we review and analyze previous studies that have investigated the measurement and implications of soft-tissue thickness (STT) on the face, particularly in relation to gender, age, BMI, and surgical outcomes. We consider the methodologies, results, and conclusions drawn in these studies to provide context and support for our own research.

\subsection{Study 1: Measurement of Facial Soft Tissue Thickness Using CBCT in Relation to BMI, Age, and Sex}

\cite{b1} conducted a study to measure facial soft tissue thickness using cone beam computed tomography (CBCT) and investigated the relationships with BMI, age, and sex. The study involved 82 patients and identified that BMI had the most significant impact on STT at various craniometric points, while age and sex had lesser influences. Specific points such as the mid-philtrum and prosthion showed substantial variations in thickness correlated with BMI, suggesting that higher BMI generally leads to increased STT in these regions. The findings highlight the importance of considering BMI in pre-operative assessments and surgical planning for facial procedures.

\subsection{Study 2: Effect of Sex and Surgical Intervention on Facial Soft Tissue Thickness}

This study, \cite{b2}, examines the impact of surgical interventions on STT, with a particular focus on differences between male and female patients. The research found that surgical procedures can significantly alter STT, and these changes may differ by sex. For instance, male patients generally exhibited thicker soft tissues post-surgery compared to females. This study emphasizes the necessity of gender-specific considerations in surgical planning and outcomes evaluation.

\subsection{Study 3: Comprehensive Analysis of Soft Tissue Thickness in Different Demographics}

The third study, \cite{b3}, utilized DICOM data to evaluate STT in a diverse patient population. The analysis revealed that demographic factors such as age, sex, and BMI significantly influence STT. The study used advanced statistical methods to compare these variables across various facial landmarks, demonstrating that STT increases with age and BMI but varies significantly between sexes. These insights are critical for personalized medical and surgical approaches, ensuring that treatments are tailored to individual patient characteristics.


\section{MATERIALS AND METHODS}

\indent

In this study, we utilized high-resolution DICOM data to measure the soft-tissue thickness (STT) at specific facial landmarks. The data collection process involved several key steps:

\subsection{Data Collection}

The primary dataset used for this research was sourced from the publicly available collection on Academic Torrents. This dataset is a subset of CT scans originally compiled for a study on head-and-neck cancer. Specifically, the dataset includes FDG-PET/CT and radiotherapy planning CT imaging data for 298 patients from four institutions in Québec, with histologically proven head-and-neck cancer. These patients had pre-treatment FDG-PET/CT scans between April 2006 and November 2014, with dates de-identified but intervals preserved. More details about this dataset can be found in the publication by Vallières et al. (2017) \cite{b4}.

\begin{itemize}
    \item URL: \url{https://academictorrents.com/details/d06aafd957f0c8c9b0eb4636e5c3ebdb7bdaf54f}
    \item License: Creative Commons Attribution 3.0 Unported License
\end{itemize}

Additionally, we analyzed data from 15 patients provided by Professor Olivetti and Professor Marcolin. These patients were distributed evenly across three hospitals, with five patients from each institution. 

\subsection{Landmarks Selection}

For each patient, the following facial landmarks were identified and measured:

\begin{itemize}
    \item Glabella
    \item Nasion
    \item Orbitale (right and left)
    \item Superius (right and left)
    \item Zygion (right and left)
    \item Rhinion
    \item A-point
\end{itemize}

These landmarks were chosen based on their relevance in craniofacial studies and their ease of identification on DICOM images.

\subsection{Measurement Procedure}

The measurements were performed on high-resolution DICOM images using advanced imaging software. The specific landmarks on the face were identified based on established fiducial points commonly used in craniofacial studies. Each landmark's STT was measured and recorded for statistical analysis.

\subsection{Statistical Analysis}

Statistical analyses were conducted to evaluate the correlation between specific facial landmarks, gender, and age of the patients. The analysis aimed to determine which facial landmarks are most indicative of the patient's gender and age. Methods such as correlation analysis and regression analysis were employed to identify significant relationships between these variables. The results were then compared with current standards in the literature to validate accuracy and reliability.

The methodology described ensures precise and reproducible measurements, which are crucial for the study's objectives. By utilizing robust statistical tools, the study aims to provide comprehensive insights into the variations in soft-tissue thickness and its correlation with gender and age, contributing valuable knowledge to the fields of surgical planning, forensic analysis, and anthropometric studies.



\section{RESULTS}

\indent

The results section should present the findings of the study clearly and concisely. This may include text, tables, and figures that summarize the data collected. Authors should highlight the most important results and provide any relevant statistical analyses. Each table and figure should be accompanied by a descriptive caption.

% Example table and figure
\begin{table}
\caption{Example table with results.}
\vskip.25cm
\centerline{\begin{tabular}{cc}
\hline
Parameter & Value \\
\hline
Parameter 1 & 10 \\
Parameter 2 & 20 \\
\hline
\end{tabular}}
\label{example_table}
\end{table}

\begin{figure}
\centerline{\includegraphics[width=0.5\columnwidth, keepaspectratio]{fig1.eps}}
\caption{This is an example of a result figure.}
\label{example_figure}
\end{figure}

\section{DISCUSSION}

\indent

In the discussion section, authors should interpret their results and explain the implications of their findings. This is where they can compare their results with those of other studies, discuss any limitations of their study, and suggest possible directions for future research. The discussion should provide a clear link between the results and the broader context of the field.

\section{CONCLUSIONS}

\indent

The conclusions section should summarize the main findings of the study and their significance. Authors should restate the purpose of the research, highlight the key results, and suggest any practical applications or implications. This section should be concise and focused on the most important takeaways from the study.

\section{ACKNOWLEDGEMENTS}

\indent

The authors would like to express their sincere gratitude to all the professors of the course for their invaluable guidance and support throughout this project:

\begin{itemize}
    \item Marcolin Federica
    \item Innocente Chiara
    \item Marullo Giorgia
    \item Montrucchio Bartolomeo
    \item Olivetti Elena Carlotta
    \item Ulrich Luca
\end{itemize}

Their expertise and encouragement have been instrumental in the successful completion of this research.


\section{REFERENCES}
{\small
\bibliographystyle{aa_url_saj}
\bibliography{references}
}

\end{document}
