\documentclass{article_saj}
\pagestyle{myheadings}
\usepackage{graphicx,saj,multicol,subeqnarray}
\usepackage{natbib}
\usepackage{float}
\usepackage{xcolor}
\usepackage{widetext}
\usepackage{url}
\usepackage{bm}
\usepackage{tikz} % for checkmark
\usepackage{pifont} % for xmark
\usepackage{amsfonts}
\usepackage{amssymb}
\usepackage{amsmath,upgreek}
\usepackage{titlesec}
\def\tg{\mathop{\rm tg}\nolimits}
\def\arctg{\mathop{\rm arctg}\nolimits}

\def\point#1{\hbox{\setbox7=\hbox to0.6em{\hfil.\hfil}%
\setbox8=\hbox to0.5em{\hfil$^{#1}$\hfil}%
\box7\kern-0.5em\box8}}

\def\pointmin#1{\hbox{\setbox2=\hbox to0.8em{\hfil.\hfil}%
\setbox3=\hbox to0.6em{\hfil$^{#1}$\hfil}%
\box2\kern-.7em\box3}}

%for non integer numbers for years, days, hours, minutes and seconds

\def\yyy{\point{\mathrm{y}}}
\def\ddd{\point{\mathrm{d}}}
\def\hhh{\point{\mathrm{h}}}
\def\mmm{\pointmin{\mathrm{m}}\kern.15em}
\def\sss{\point{\mathrm{s}}}

%for non integer numbers for arc degrees, arc minutes and arc seconds

\def\oo{\point{\circ}}
\def\lll{\point{\prime}}
\def\uu{\point{\prime\prime}}

%for integer numbers for arc degrees, arc minutes and arc seconds

\def\OO{$^\circ$}
\def\LLL{$^\prime$}
\def\UU{$^{\prime\prime}$}
 

\titlelabel{\thetitle.\quad}
\definecolor{xlinkcolor}{cmyk}{1,0.6,0,0}
\usepackage[bookmarks=false,         % show bookmarks bar?
     pdfnewwindow=true,      % links in new window
     colorlinks=true,    % false: boxed links; true: colored links
     linkcolor=xlinkcolor,     % color of internal links
     citecolor=xlinkcolor,     % color of links to bibliography
     filecolor=xlinkcolor,  % color of file links
     urlcolor=xlinkcolor,      % color of external links
final=true
]{hyperref}

% Papertype can be "Invited Review", "Original Scientific Paper",
% "Preliminary report" or "Professional paper"

\def\papertype{\ \hfill\ Editorial}

\def\udc{52}
\setcounter{publno}{200}
\setcounter{publyear}{2020}
\setcounter{page}{1}
\setcounter{firstpage}{1}
\setcounter{lastpage}{5}

\citestyle{kluwer}%

\setcounter{footnote}{0}
\renewcommand{\thefootnote}{\fnsymbol{footnote}}

\begin{document}
%
\parindent=.5cm
\baselineskip=3.8truemm
\columnsep=.5truecm
%
\newenvironment{lefteqnarray}{\arraycolsep=0pt\begin{eqnarray}}
{\end{eqnarray}\protect\aftergroup\ignorespaces}
\newenvironment{lefteqnarray*}{\arraycolsep=0pt\begin{eqnarray*}}
{\end{eqnarray*}\protect\aftergroup\ignorespaces}
\newenvironment{leftsubeqnarray}{\arraycolsep=0pt\begin{subeqnarray}}
{\end{subeqnarray}\protect\aftergroup\ignorespaces}
%

% Runningtitle

\markboth{\eightrm INSTRUCTIONS FOR AUTHORS} 
{\eightrm B. ARBUTINA {\lowercase{\eightit{et al.}}}}

\begin{strip}

{\ }

\vskip-1cm

\publ

\type

{\ }

% Title

\title{INSTRUCTIONS FOR AUTHORS}

% Authors

\authors{B. Arbutina$^{1}$, D. Uro{\v s}evi{\' c}$^{1}$ and M. Jovanovi{\' c}$^2$}

\vskip3mm

% Address

\address{$^1$Department of Astronomy, Faculty of Mathematics,
University of Belgrade\break Studentski trg 16, 11000 Belgrade,
Serbia}

% E-mail

\Email{arbo@math.rs, dejanu@math.rs}

\address{$^2$Astronomical Observatory, Volgina 7, 11060 Belgrade 38, Serbia}

\Email{milena@aob.rs}

% Received and Accepted dates

\dates{May 18, 2020}{June 1, 2020}

% Abstract

\summary{The purpose of this editorial is to help authors in
preparing papers to be submitted to our journal. Papers should be
prepared in \LaTeX\ by using Serbian Astronomical Journal style
file which has some predefined fields in the preamble and some
additional or slightly modified commands for the main text. The
authors are welcome to use this document as the sample paper and/or template.}

% Keywords (see keywords.pdf file)

\keywords{Editorials notices -- Publications, bibliography --
Miscellaneous}

\end{strip}

\tenrm

% Sections

\section{INTRODUCTION}

\indent

This\footnotetext[0]{\copyright \ 2020 The Author(s). Published by
Astronomical Observatory of Belgrade and Faculty of Mathematics,
University of Belgrade. This open access article is distributed under
CC BY-NC-ND 4.0 International licence.}
editorial is meant to provide help to authors preparing
papers to be submitted to our journal. Serbian Astronomical
Journal (SerAJ) is published semiannually (in June and December)
by the Astronomical Observatory and the Department of Astronomy,
Belgrade, Serbia. It publishes invited reviews, original
scientific papers, preliminary reports, and professional papers
over the entire range of astronomy, astrophysics, astrobiology and
related fields. 

\renewcommand{\thefootnote}{\arabic{footnote}}

Papers should be prepared in \LaTeX\ by using SerAJ style file
which has some predefined fields in the preamble (paper category,
title, authors, affiliations and e-mails, keywords and summary)
and some additional or slightly modified commands for the main
text. The authors are welcome to use this document as the sample
paper and/or template. Comparing its \LaTeX\ source with the output it generates can show one how
to produce a simple document of his/her own. Summary should
contain at maximum 300 words (but much less is desirable) in
order for the abstract to fit entirely on the first page. Authors have
to use keywords from the list which is common to the major
astronomical and astrophysical journals. The list is available at
journal's website: \url{http://saj.math.rs/index.php?id=key}.

Submission to our journal is a representation that the manuscript
has not been published elsewhere. Each article submitted for
publication is peer-reviewed. There are no paper charges. If
usable color figures have been submitted with article, they will
appear in color on the web, in the electronic edition of the
journal. In the printed issue, color reproduction depends on
journal policy and will be decided on a case-by-case basis.

Publication in the Serbian Astronomical Journal consists of
several steps:

\parindent=.7cm

\par\hang\textindent{(i)} submission to the journal (more detailed instructions are available upon
request),

\par\hang\textindent{(ii)} referee is chosen by the editors and we are
waiting for his report,

\par\hang\textindent{(iii)} referee's report has been received and (minor or major)
revision is required (note that this step may be repeated couple
of times), or acceptance/rejection of the paper has been
suggested. If revised paper has not been received within 3 months it will be
considered withdrawn.

\par\hang\textindent{(iv)} If paper has been accepted, before publication author
will receive a proof of his/her article for small corrections. The
proof is used solely for checking the typesetting and editing,
also the completeness and correctness of the text, tables, and
figures. Changes to the article as accepted for publication will
not be considered at this stage.

\par\hang\textindent{(v)} Paper has been published and it is available electronically at SerAJ website and
shortly after in hard copy, as well
as in the selected bibliographic services.

\parindent=.5cm

Serbian Astronomical Journal is currently indexed or abstracted in
Astrophysics Data System (ADS),
Clarivate Analytics' Web of Science and Journal Citation Report,
Scopus, Chemical Abstracts, Referativni Zhurnal, EBSCO, SRJ SCImago,
DOAJ, Serbian Citation Index (SCIndeks), DOISerbia.

\section{INSTRUCTIONS}

\indent

In the following subsections we will give some examples how to
typeset mathematical formulas, include figures and tables and
correctly prepare references. Our intention is not to cover all possible aspects
of preparing a scientific text -- for more detailed instructions
one should consult some of the many \LaTeX\ manuals.

Here is just an example of quotation:

\vskip.2cm

\noindent {\nineit Astronomy is useful because it raises us above
ourselves; it is useful because it is grand$\dots$ It shows us how
small is man's body, how great his mind, since his intelligence
can embrace the whole of this dazzling immensity, where his body
is only an obscure point, and enjoy its silent harmony.}
\vskip.1cm

\rightline{\ninerm Henri Poincar\'e (1854 -- 1912)}

% Subsections

\subsection{Equations, figures, tables}

\indent

\LaTeX\ is good at typesetting mathematical formulas or symbols
such as $\alpha$, $\beta$, $\Gamma$, $\Sigma$, $\pi$, $\nabla$, $\partial$, $\forall$, $\exists$, $\times$, $\pm$, $\infty$ etc.
Equations should be enumerated.

% Equations

The following is an example of equation:
\begin{equation}
\Sigma=AD^{-\beta}.
\label{eq:1}
\end{equation}

% "\begin{equation}" can be replaced with "$$"
% "\end{equation}" can be replaced with "$$"
% In that case "\label{(number)}" may be replaced with "\eqno{(number)}"

One can also use equation arrays:
\begin{lefteqnarray}
\label{eq:f}
&& f(x)=\log(x^2+3x), \\
\label{eq:g}
&& g(x)=x+\sin x,
\end{lefteqnarray}
or
% SUBEQUATIONS
subequations (i.e. subequation arrays):
\begin{leftsubeqnarray}
\slabel{eq:casa}
&& \sin\theta=0.24, \\
\slabel{eq:casb} && \cos\alpha=0.47. \label{seq:cas}
\end{leftsubeqnarray}
In the text equations should be quoted Eq. (\ref{eq:1}) or Eqs. (\ref{eq:f}) and (\ref{eq:g}).

% \slabel{eq:namei} recalls the single equation e.g. (2i) (i=a,b,...)
%
% \label{seq:name}  recalls the whole group of subequations e.g., (2)

% Tables

Tables \ref{table1} and \ref{table2} are examples of tables. For large tables it is
necessary to switch from two column to single column environment.
Table \ref{table1} gives number of students graduated from Department of
Astronomy, Faculty of Mathematics, University of Belgrade in a
given period. For more information check Belgrade astronomical
community database -- BAZA (from Serbian: Beogradska Astronomska
ZAjednica) at \url{http://alas.matf.bg.ac.rs/\~astrobaza/} (see
also \citeauthor{2009ata} \citeyear{2009ata}).

% Onecolumn table

\begin{table}
\caption{{An example of a small table (number of students
graduated from Department of Astronomy, Faculty of Mathematics,
University of Belgrade in a given period).}}
\vskip.25cm
\centerline{\begin{tabular}{cc}
\hline
Period & Graduated students \\
\hline
1935/36 -- 1939/40 & 2 \\
1940/41 -- 1949/50 & 2 \\
1950/51 -- 1959/60 & 31 \\
1960/61 -- 1969/70 & 20 \\
1970/71 -- 1979/80 & 38 \\
1980/81 -- 1989/90 & 35 \\
1990/91 -- 1999/00 & 51 \\
2000/01 -- 2009/10 & 77 \\
\hline
\end{tabular}}
\label{table1}
\end{table}

% Twocolumn table

\begin{table*}
\caption{An example of a large table (citation of the
Serbian Astronomical Journal in the period 2003--2010).}
\parbox{\textwidth}{
\vskip.25cm
\centerline{\begin{tabular}{@{\extracolsep{0.0mm}}lcccccccccccc@{}}
  \hline
    Year ($y$)  & \multicolumn{6}{c}{No. of articles in} & \multicolumn{4}{c}{Cites to articles from} &   \multicolumn{2}{c}{Impact factor} \\
              & \multicolumn{3}{c}{$y-1$} & \multicolumn{3}{c}{$y-2$} & \multicolumn{2}{c}{$y-1$} & \multicolumn{2}{c}{$y-2$} & \multicolumn{2}{c}{\ } \\
              &&  &&&  && {\footnotesize Other cites} & {\footnotesize Self cites} & {\footnotesize Other cites} & {\footnotesize Self cites} & &   \\
 \hline
 2003         && 8 &&& 16 && 0 & 3 & 2 & 2 & 7/24 &  0.292\\
 2004         && 26 &&& 8 && 3 & 9 & 1 & 0 & 13/34 &  0.382\\
 2005        && 17 &&& 26 && 5 & 4 & 5 & 0 & 14/43 &  0.326\\
 2006        && 21 &&& 17 && 2 & 4 & 1 & 2 & 9/38 &  0.237\\
 2007        && 19 &&& 21 && 2 & 6 & 6 & 2 & 16/40 &  0.400\\
 2008        && 18 &&& 19 && 5 & 4 & 10 & 5 & 24/37 &  0.649\\
 2009        && 23 &&& 18 && 15 & 11 & 7 & 2 & 35/41 &  0.854\\
 2010        && 21 &&& 23 && 5 & 3 & 8 & 3 & 19/44 &  0.432\\
   \hline
\end{tabular}}}
\label{table2}
\end{table*}

% Figures in ps or eps format with resolution 600dpi

% Onecolumn figure

\begin{figure}
\centerline{\includegraphics[width=0.5\columnwidth, keepaspectratio]{fig1.eps}}
\caption{This is an example of a small figure.}
\label{fig1}
\end{figure}

% Twocolumn figure

\begin{figure*}[ht!]
\centerline{\includegraphics[width=0.73\textwidth, keepaspectratio]{fig2.eps}}
\caption{This is an example of a large figure. It is
necesary to switch from two column to single column environment.}
\label{fig2}
\end{figure*}

Table \ref{table2} shows the number of articles published in SerAJ, number of
cites to articles published in SerAJ (self cites and other cites)
and calculated impact factors (IF) for years
2003--2010.\footnote{More information about SerAJ citation can be
found in \citet{2007Arb,2010Arb,2013Arb}.}

Figs. \ref{fig1} and \ref{fig2} are examples of small and large figure,
respectively. Figures should be in ps or eps format with
recommended resolution 600 dpi (300 dpi minimum). Figure captions should be
below the figure itself. As with the tables, for large figures it
is necessary to switch from two column to single column
environment, and then switch back to two columns afterwards.

% Subsections

\subsection{Citations and references}

\indent

Make sure that all the citations that appear in the text are listed
correctly in the References, ordered alphabetically by surname
(with initials following). If there are several references to the
same first author, they should be entered according to the
following scheme:

\parindent=.7cm

\par\hang\textindent{(1)} one author: chronologically,

\par\hang\textindent{(2)}  author and one co-author: alphabetically by co-author, then
chronologically,

\par\hang\textindent{(3)} author, two or more co-authors: chronologically.

\parindent=.5cm

You can cite an author directly in the text, such as \citet{1986Ber},
or after some statement, in parentheses separated by
commas \citep{2020Doe,1978Bel,1983All,1987Bin,1998Cas,2004Uro,2005Cir,2007Met,2008Ive,2020Smith,
2015aste.book..297N}. Use
'et al.' in the text for more than two authors, and in the
references for more than eight authors (the first three being named). If more than one citation
for a particular author (author group) is made for the same year,
'a', 'b', 'c', $\ldots$ should be added to the year. Do not
include the title in a reference, except for books and monographs
that are not published in periodical publications. If citation does
not correspond to any publication, after the author's name and
year one usually puts: 'in preparation' or 'private communication'.
If paper has not been published yet and there is no ArXiv preprint
either, instead of the full reference with pagination one can put
journal's title followed by one of these statements: 'submitted',
'accepted', 'in press', or put DOI number if available. Use short (ADS)
journals abbreviations when preparing references (e.g.
SerAJ).\footnote{A list of journals in astronomy and
astrophysics, space sciences and related fields, together with its
abbreviations is available at
\url{http://saj.math.rs/index.php?id=abbr}.}

% Acknowledgements

\acknowledgements{This is the place for acknowledgements such as:
during the work on this paper the authors were financially
supported by the Ministry of Education and Science of the Republic
of Serbia through the projects:  176004 'Stellar physics',
176005 'Emission nebulae: structure and evolution', 176011 'Kinematics and
dynamics of celestial bodies and systems'.}

% References

%\vskip2mm

\newcommand\eprint{in press }

\bibsep=0pt

\bibliographystyle{aa_url_saj}

{\small

\bibliography{sample_saj}
}

\begin{strip}

\end{strip}

\clearpage

{\ }

\clearpage

{\ }

\newpage

\begin{strip}

{\ }

% Serbian abstract

% Title

\naslov{UPU{T}{S}TVO ZA AUTORE}

% Authors

\authors{B. Arbutina$^{1}$, D. Uro{\v s}evi{\' c}$^{1}$ and M. Jovanovi{\' c}$^2$}

\vskip3mm

% Address

\address{$^1$Department of Astronomy, Faculty of Mathematics,
University of Belgrade\break Studentski trg 16, 11000 Belgrade,
Serbia}

% E-mail

\Email{arbo@math.rs, dejanu@math.rs}

\address{$^2$Astronomical Observatory, Volgina 7, 11060 Belgrade 38, Serbia}

\Email{milena@aob.rs}

\vskip3mm

% UDC

\centerline{{\rrm UDK} \udc}

% Papertype

\vskip1mm

\centerline{\rit Uredjivaqki prilog}

\vskip.7cm

\baselineskip=3.8truemm

\begin{multicols}{2}

{
\rrm

Ovaj prilog ima za cilj da pomogne autorima kod pripreme
qlanaka za na{\ss} qasopis.~Radove, po mogu{\cc}nosti, treba
pripremiti koriste{\cc}i}  \LaTeX\ {\rrm uz posebnu datoteku koja
defini{\ss}e izgled qasopisa} Serbian Astronomical Journal {\rrm i
sadr{\zz}i unapred definisana polja u preambuli dokumenta i neke
dodatne ili de\-li\-miq\-no izmenjene komande koje se koriste pri unosu
samog teksta.~Autori mogu koristiti ovaj dokument kao primer pri
kucanju svojih radova.

{\ }

}

\end{multicols}

\end{strip}

%%%%%%%%%%%%%%%%%%%%%%%%%%%%%%%%%%%%%%%%%%%%%%%%%%%%%%%%%%%%%%%%%%%%%%%%%%%%%
%                       C O M P I L I N G
% You must compile your LaTeX source text "mysource.tex" in THE following
% way:
%  pdflatex mysource
%  bibtex mysource
%  pdflatex mysource
%  pdflatex mysource
% Compiling twice is required to set all the parameters needed
% (lastpage, labels...) properly. It will also generate the mysource.bbl
% with your bibliography.
%%%%%%%%%%%%%%%%%%%%%%%%%%%%%%%%%%%%%%%%%%%%%%%%%%%%%%%%%%%%%%%%%%%%%%%%%%%%%

\end{document}
